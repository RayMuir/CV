% https://www.ctan.org/tex-archive/macros/latex/contrib/moderntimeline
% https://www.ctan.org/tex-archive/macros/latex/contrib/moderncv


\documentclass[12pt, a4paper, sans]{moderncv}
\moderncvstyle{classic}
\moderncvcolor{blue} 
\usepackage[utf8]{inputenc}
\usepackage[scale=0.85]{geometry}    % Width of the entire CV
\setlength{\hintscolumnwidth}{3cm} % Width of the timeline on your left
\usepackage{pdfpages/pdfpages}
\usepackage{moderntimeline/moderntimeline}
\usepackage{xpatch/xpatch}
\usepackage{color, graphicx}
\tlmaxdates{2016}{2024}              % Beggining and start of your timeline                        

\newcommand{\cvreferencecolumn}[2]{%
  \cvitem[0.8em]{}{%
    \begin{minipage}[t]{\listdoubleitemmaincolumnwidth}#1\end{minipage}%
    \hfill%
    \begin{minipage}[t]{\listdoubleitemmaincolumnwidth}#2\end{minipage}%
    }%
}

\newcommand{\cvreference}[8]{%
    \textbf{#1}\newline% Name
    \ifthenelse{\equal{#2}{}}{}{\addresssymbol~#2\newline}%
    \ifthenelse{\equal{#3}{}}{}{#3\newline}%
    \ifthenelse{\equal{#4}{}}{}{#4\newline}%
    \ifthenelse{\equal{#5}{}}{}{#5\newline}%
    \ifthenelse{\equal{#6}{}}{}{\emailsymbol~\texttt{\href{mailto:#6}{\nolinkurl{#6}}}\newline}%
    \ifthenelse{\equal{#7}{}}{}{\phonesymbol~#7\newline}
    \ifthenelse{\equal{#8}{}}{}{\mobilephonesymbol~#8}}

% Personal Information
\name{Rachel Muir}{}                             
\phone[mobile]{+44~(0)~7483~808~358}                   
\email{rm717@kent.ac.uk}               
\quote{Durum Patientia Frango}                       
\makeatother


\begin{document}

\makecvtitle

\section{Educational Background}

\bigskip

\medskip
\tlcventry{2023}{0}{PhD in Theoretical Computer Science}{Current}{}{}{\cvitem{\textbf{Advisors:}}{Prof Mark Batty \& Dr Michael Vollmer}}
\medskip
\tlcventry{2019}{2023}{BSc (Hons)}{University of Kent, Canterbury}{}{}{Strong foundation in Java,
OCaml, Python and C++ programming skills, writing algorithms, testing and refining code, and using
version control software.
\\
Theory of Computing combined maths, modelling systems and logical semantics to describe the
mechanisms behind systems, and proofs to confirm their behaviour. 
\\
Internet of Things introduced me to basic electronics, development kits with development
environments, and capturing and processing data through multiple devices communicating. This enabled
me to begin designing my own bespoke PCBs containing circuitry I created, code software to utilise
microcontrollers, and deep diving to low level architecture to create my own versions of various
development boards. To this day, I still create my own bespoke circuits and software.
\\
Computer Systems led to valuable insight about low-level architecture, assembly, and high-level
abstract structures. Programming Language Implementation built upon this to enrich my knowledge
covering compilers. Understanding the multitude of complex sections that build a complier through
extending the parser, AST, CFG and Optimizer has been invaluable for my research in software
verification (x86) through formal semantics. 
}
\medskip

\tlcventry{2022}{2023}{Formal Verification of a MIR-to-MIR Optimisation}{}{}{}{
\cvitem{\textbf{Advisor:}}{Professor Mark Batty} 
\medskip
The goal of my final year project was to provide a first step
to proving the soundness of Rust's middle intermediate representation (MIR), starting with a modest
optimisation. This required a grammar that represented a set of programs that the optimisation
targets, a formal semantics and a formal definition of the optimisation and applying the semantics
and definitions to prove the soundness of simplify branches. The project was shortlisted for the
PLDI Student Research Competition and now forms the basis of my PhD.}





\bigskip

\tlcventry{2016}{2019}{College Qualifications}{Bexhill College}{}{}{}
\cvitemwithcomment{Applied Science}{180 Credit Extended Diploma}{\emph{Distinction* Distinction* Distinction*}}

\cvitemwithcomment{Information Technologies}{90 Credit Certificate}{\emph{Distinction* Distinction*}}

\cvitemwithcomment{Mathematical Studies}{Certificate}{\emph{B}}

\cvitemwithcomment{7 GCSEs}{Grades 7-C}{\emph{Maths 7, English B}}




\pagebreak
\section{Work Experience}
\tlcventry{2019}{2022}{Academic Ambassador}{University of Kent}{Canterbury}{}{}
% {Working for the central university and School of Computing has involved demonstrating the
% advantages of further education (FE) and giving insight into student life through on-campus tours
% and virtual/physical open days such as the First Lego League Challenge. I liaise with teaching staff
% at outreach schools, and teach secondary school to college age students with regards to FE, it's
% processes and what to expect. I adapt my communication style accordingly when delivering
% presentations to up to 30 students. I further developed my skills and confidence by designing and
% delivering succinct, polished presentations.}

\medskip

\tlcventry{2021}{2022}{Developer and Maintainer}{Smart Start Minds}{London - Remote}{}{In my second
year of University, I was invited into a start-up research firm to join a team of developers which
research and develop bespoke hardware and software to improve and treat mental health conditions.
Our neurofeedback hardware, in the form of non-invasive imaging (fNIRS), measures changes in the
concentration of oxygenated haemoglobin. I designed a prototype web-based (offline and online)
system to be used globally, alongside our hardware, to detect lulls in mental concentration. The
user is prompted to perform an activity and increasing concentration to above baseline. This
world-leading treatment allows conscious alteration of the nervous system at an affordable cost.
The prototype enabled the company to demonstrate the viability of an automated self-treatment
method to businesses and funders interested in using or developing this technology. More
importantly, it demonstrated accessible and affordable treatment globally, by allowing treatment of
patients in low socio-economic areas and/or patients without access to in-person healthcare,
internet, or medical funding.}

\medskip

\tldatecventry{2022}{Rolls-Royce Data Scientist}{R\textsuperscript{2} Data Labs}{London -
Hybrid}{}{During my time at R\textsuperscript{2} Data Labs, I was apart of a team that developed
software for customers. This involved programming in python, learning data science methods and
libraries, daily stand-ups, project management with GitHub, communicating requirements with the
customer and getting feedback, and following an agile work method. Our projects consisted of making
NLP based software to aid the customer in completing required tasks, and presenting that data in a
human-readable way. We worked through prototyping stages over multiple weeks and developed the
software into minimum-viable-product, which was then delivered to the customer.}

% \medskip

% \tldatecventry{2023}{Research Assistant}{The University of Kent}{In-Person}{}{}

\medskip

\section{Prizes}
\tldatecventry{2023}{Student Research Competition}{Shortlisted}{PLDI}{}{}
\tldatecventry{2022}{Undergraduate of the Year 2022}{Winner}{Target Jobs}{}{}
\tldatecventry{2022}{Kent Star}{Winner}{The University of Kent}{}{}


\section{Conferences}
\tldatecventry{2020}{36th Chaos Communication Congress}{}{Technology and Cyber Security Conference}{Leipzig}{}
\tldatecventry{2023}{PLDI}{Florida}{Invited - Student Research Competition}{Poster Session}{}

\section{Summer Schools}
\tldatecventry{2023}{OPLSS}{}{University of Oregon}{}{}
\tldatecventry{2023}{VetSS}{}{University of Sussex}{}{}


\section{References}
\cvitemwithcomment{Prof Mark Batty}{M.J.Batty@kent.ac.uk}{}
\cvitemwithcomment{Dr Michael Vollmer}{M.Vollmer@kent.ac.uk}{}
\clearpage

\end{document}