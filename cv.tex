% https://www.ctan.org/tex-archive/macros/latex/contrib/moderntimeline
% https://www.ctan.org/tex-archive/macros/latex/contrib/moderncv


\documentclass[12pt, a4paper, sans]{moderncv}
\moderncvstyle{classic}
\moderncvcolor{blue} 
\usepackage[utf8]{inputenc}
\usepackage[scale=0.85]{geometry}    % Width of the entire CV
\setlength{\hintscolumnwidth}{3cm} % Width of the timeline on your left
\usepackage{pdfpages/pdfpages}
\usepackage{moderntimeline/moderntimeline}
\usepackage{xpatch/xpatch}
\usepackage{color, graphicx}
\tlmaxdates{2016}{2025}        % Beggining and start of your timeline                        

\newcommand{\cvreferencecolumn}[2]{%
  \cvitem[0.8em]{}{%
    \begin{minipage}[t]{\listdoubleitemmaincolumnwidth}#1\end{minipage}%
    \hfill%
    \begin{minipage}[t]{\listdoubleitemmaincolumnwidth}#2\end{minipage}%
    }%
}

\newcommand{\cvreference}[8]{%
    \textbf{#1}\newline% Name
    \ifthenelse{\equal{#2}{}}{}{\addresssymbol~#2\newline}%
    \ifthenelse{\equal{#3}{}}{}{#3\newline}%
    \ifthenelse{\equal{#4}{}}{}{#4\newline}%
    \ifthenelse{\equal{#5}{}}{}{#5\newline}%
    \ifthenelse{\equal{#6}{}}{}{\emailsymbol~\texttt{\href{mailto:#6}{\nolinkurl{#6}}}\newline}%
    \ifthenelse{\equal{#7}{}}{}{\phonesymbol~#7\newline}
    \ifthenelse{\equal{#8}{}}{}{\mobilephonesymbol~#8}}

% Personal Information
\name{Rachel Muir}{}                             
% \phone[mobile]{+44~(0)~7483~808~358}                   
\email{rm717@kent.ac.uk}               
% \quote{Durum Patientia Frango}                       
\makeatother


\begin{document}

\makecvtitle
\vspace{-2em}
\section{Education}

\bigskip

\medskip
\tlcventry{2023}{0}{PhD, Programming Languages and Systems}{Current- Year Two}{}{}{\cvitem{\textbf{Advisors:}}{Prof Mark Batty \& Dr Michael Vollmer -- University of Kent, PLAS}
\medskip
Current work involves extending G.Morrisett and D. Walker's typed assembly
language (TAL). We add vector instructions to the language extending the type
system and proofs accordingly. My core contributions are designing a grammar,
type system and operational semantics to capture the behaviour of vector
instructions. I have implemented these semantics in OCaml with the goal of
formally proving the correctness of our semantics in Rocq.}

\medskip


\tlcventry{2019}{2023}{BSc (Hons)}{University of Kent, Canterbury}
{}{}{
\cvitem{\textbf{Dissertation:}}{Formal Verification of a MIR-to-MIR Optimisation}}
% \cvitem{\textbf{Advisor:}}{Professor Mark Batty}
% \medskip
% This dissertation aimed to provide a first step towards proving the soundness of
% Rust's middle intermediate representation, starting with a modest optimisation (tautology elimination).
% It required defining a grammar and formal semantics, a formal definition of a
% target optimisation, and using proof techniques over the semantics to prove their soundness. This project was shortlisted for the PLDI Student
% Research Competition. % Ask Mark about this last sentence


% \textbf{Languages:} Strongest languages are OCaml and Python. Strong
% foundation in Java. Basic C++ programming skills. \\
% \textbf{Theory of
% Computing:} Propositional logic, first-order logic, natural deduction, DPLL,
% DFA \& NFA, CFGs, Recursive Decent Parsing. \\
% \textbf{Computer Systems:} Boolean
% algebra, logic gates, sequential logic, computer architecture i.e. PC, CPU,
% ALU, GPU, parsing, virtual machines. \\
% \textbf{Programming Language
% Implementation:} OCaml basics, compiler architecture, AST, CFG, optimisations,
% register allocation. \\
% \textbf{IoT \& Embedded Computer Systems:} Low-level
% programming, parallelism, RTOS, circuit design, MBED/ESP32 microcontrollers.
% }

% {Formal Verification of a MIR-to-MIR Optimisation}
% {\cvitem{\textbf{Advisor:}}{Professor Mark Batty}
% \medskip
% This project aimed to provide a first step towards proving the soundness of
% Rust's middle intermediate representation, starting with a modest optimisation.
% It required defining a grammar and formal semantics, a formal definition of a
% target optimisation, and using proof techniques over the semantics to prove the
% soundness of an optimisation. This project was shortlisted for the PLDI Student
% Research Competition.}{}

\medskip

\tlcventry{2016}{2019}{College Qualifications}{Bexhill College}{}{}{}
\cvitemwithcomment{Applied Science}{180 Credit Extended Diploma}{\emph{Distinction* Distinction* Distinction*}}

\cvitemwithcomment{Information Technologies}{90 Credit Certificate}{\emph{Distinction* Distinction*}}

% \cvitemwithcomment{Mathematical Studies}{Certificate}{\emph{B}}

% \cvitemwithcomment{7 GCSEs}{Grades 7-C}{\emph{Maths 7, English B}}

% \section{Research}


% \newpage

\section{Work Experience}
\tldatecventry{2023}{Summer Research Assistant}{The University of Kent}{In-Person}{}
{This work aimed to provide a first step towards proving the soundness of Rust's
middle intermediate representation (MIR), starting with a modest optimisation
(tautology elimination). It required defining a grammar and formal semantics, a
formal definition of a target optimisation, and using proof techniques over the
semantics to prove their soundness. This project was shortlisted for the PLDI
Student Research Competition.}

\medskip

\tlcventry{2019}{2022}{Academic Ambassador}{University of Kent}{Canterbury}{}{}
% {Working for the central university and School of Computing has involved demonstrating the
% advantages of further education (FE) and giving insight into student life through on-campus tours
% and virtual/physical open days such as the First Lego League Challenge. I liaise with teaching staff
% at outreach schools, and teach secondary school to college age students with regards to FE, it's
% processes and what to expect. I adapt my communication style accordingly when delivering
% presentations to up to 30 students. I further developed my skills and confidence by designing and
% delivering succinct, polished presentations.}

\medskip

\tlcventry{2021}{2022}{Developer and Maintainer}{Smart Start Minds}{London -
Remote}{\textit{Javascript}}{In my second year of university, I joined a team of
developers in researching and developing bespoke hardware and software to
improve and treat mental health conditions. Our neurofeedback hardware, in the
form of non-invasive imaging (fNIRS), measures changes in the concentration of
oxygenated haemoglobin. I designed a web-based prototype system to detect lulls
in mental concentration and to prompt the user to perform an activity,
increasing concentration above their baseline. This prototype software enabled
the company to demonstrate the viability of this automated self-treatment method
to businesses and funders interested in using this technology. More importantly,
it demonstrated that accessible and affordable treatment could reasonably allow
treatment of patients in low socio-economic areas globally and patients without
access to in-person healthcare, internet, or medical funding. I won the UK-wide
\emph{Undergraduate of the Year} award for this work.}

\medskip

% Last 2 sentences
\tldatecventry{2022}{Rolls-Royce Data Scientist}{R\textsuperscript{2} Data
Labs}{London - Hybrid Internship}{\textit{Python}}{I was a part of a team
developing developing bespoke AI-based software. This role required knowledge of
Python, Github, and data science methods. Our projects consisted of making
NLP-based software to aid in presenting large collections of data in a
human-readable way. We worked through prototyping stages over multiple weeks and
developed the software into a minimum-viable-product, which we then delivered to
the customer. I optimised training throughput by reducing the memory footprint
of data, improving the performance of inference and adding checkpointing and
data change detection to cut downtime drastically. }

\medskip

\section{Prizes}
\tldatecventry{2022}{Undergraduate of the Year 2022}{Winner}{Target Jobs}{}{}
\tldatecventry{2022}{Kent Star}{Winner}{The University of Kent}{}{}
\tldatecventry{2023}{Student Research Competition}{Shortlisted}{PLDI}{}{}


\section{Conferences}
\tldatecventry{2020}{36th Chaos Communication Congress}{}{Technology and Cyber Security Conference}{Leipzig}{}
\tldatecventry{2023}{PLDI}{Florida}{Invited - Student Research Competition}{Poster Session}{}

\section{Summer Schools}
\tldatecventry{2023}{OPLSS}{}{University of Oregon}{}{}
\tldatecventry{2023}{VetSS}{}{University of Sussex}{}{}
\tldatecventry{2024}{Advanced Functional Programming}{}{University of Utrecht}{1.5 ECTS}{}

\section{Volunteering}
% Accessible sentences are split
\tlcventry{2022}{0}{TinkerSoc}{Committee}{\textit{TinkerSoc.org}}{}{ TinkerSoc
  is a maker society for hobby electronics, 3D printing, and engineering. I was
  president for two years, and am now vice-president. I rebuilt the society
  after COVID, by leading people through their first projects with accessible
  learning plans. This worked, drawing a more diverse, and less exclusively
  male, membership. I ran interactive seminars and training courses and built an
  amazing community. It is my proudest achievement.}


\section{References}
\cvitemwithcomment{Prof. Mark Batty}{M.J.Batty@kent.ac.uk}{}
\cvitemwithcomment{Dr. Michael Vollmer}{M.Vollmer@kent.ac.uk}{}
\clearpage

\end{document}